\section{Related Work}
\label{sec:relatedworksemblog}

Semantic blogging was introduced by \cite{Cayzer2003}, and since then has received much interest. Later \cite{Karger2004} described semantic blogging in the context of the Semantic Web with Haystack. So far, existing systems for semantic blogging fall into two categories: 
\begin{itemize}
 \item desktop applications that involve publishing the actual local resource information together with the blog post, or
 \item online applications that do not have access to desktop data relevant to the user. 
\end{itemize}

The tools in the first category, like SemBlog \cite{Takeda2005} and SemiBlog \cite{Moeller2005}, have the advantage that users have better access to the relevant data from the desktop. However, both tools require that the resources that contain sensitive private information are published together with the blog posts, which might lead to privacy issues. The SemBlog project allows users to add data from personal ontologies to their blogs. SemiBlog allows integration of personal data in the posts by drag and drop from various desktop applications like the address book. SemBlog and SemiBlog are used for exchange of personal information in the blog posts, which differs from our approach of using already published Web data as to protect the privacy of the personal information. SemiBlog's process implies manually adding the metadata, while our approach relies on automatic export. Both tools comply with our first requirement, but not with the last three.

Online services like BlogAccord \cite{Cayzer2006} for music information, and Zemanta\footnote{\url{http://www.zemanta.com}} blogging assistant, belong to the second category. They have access to various online resources to create the context of a blog post and enhance the blogging experience, but they do not use the personal context of the user.

