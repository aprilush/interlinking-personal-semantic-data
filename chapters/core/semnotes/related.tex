\section{Related Work}
\label{sec:notetakingapps}

Semantic note-taking means enhancing the note-taking process using Semantic Web technologies. It can refer to the techniques and methods used in the implementation, like ontologies and RDF, but most importantly it is about creating a semantic network around the notes and the information contained in them. There are several applications that enable more or less semantic note-taking: some are browser based (online or offline), while others are standalone desktop applications as is SemNotes. 

The List.it browser-based note-taking tool \cite{Kleek2009} and Jourknow, its predecessor \cite{Kleek2007}, save context alongside the information scraps, to improve re-finding and reminding. List.it also features information extraction from the unstructured text of the notes, recognising entities and relations between them, with the \emph{pidgin} language processor. However, unlike SemNotes, the contex of the notes they create does not include links to any existing desktop resources. SnapShoot \cite{Iga2006} is another browser-based note-taking tool that explores new visualisation techniques to improve reading of the documents produced. It features categorisation, and limited interlinking with documents within the system.

MindRaider \cite{MindRaider}, described above in Section \ref{sec:sdsystems}, is an open source ``Semantic Web outliner'' and extended note-taking system for information organisation. While it only interlinks concepts within its maps, it can connect to the Gnowsis Semantic Desktop through a plugin, thus potentially it can use any Semantic Desktop resources through a similar mechanism.

A distinct category of semantic note-taking applications are personal semantic wikis, like Kaukolu \cite{Elst2008}, IkeWiki \cite{Schaffert2006} and GDKTiddlyWiki. Each wiki page represents a resource and its semantic relations to other resources are encoded within the page, using an extension to the wiki syntax. Only predefined relations and types are available, and the wikis offer limited access to other desktop information sources. Unlike SemNotes, connections are only possible between resources within the wiki system.

OneNote from Microsoft's office suite provides quasi-se\-man\-tic functionality by interlinking the notes with address book information, calendar, and tasks. It does not use any semantic technologies though, and the data is locked in by proprietary formats and storage.

Zemanta\footnote{\url{http://www.zemanta.com}} is a blog assistant that suggests possible enhancements to blog posts, like linking external content and images. Unlike SemNotes, which uses the local repository to search for matches, Zemanta looks on the Web. It does not assign any semantics to the links. 

There are also specialised systems like the SemanticPen \cite{Varadarajan2005} which provides support for semantic note-taking with pen devices.
Other note-taking applications that provide semantic features include: Jenga Note\footnote{\url{http://www.jenganote.org}} --- allows associating a note with a concept, Catch Notes\footnote{\url{http://www.catch.com}}, and SpringPad\footnote{\url{http://www.springpadit.com}}.
