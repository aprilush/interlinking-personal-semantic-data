\section{Introduction}
\label{sec:semnotesintro}

The Semantic Desktop provides a framework for creating applications and tools that simplify the daunting tasks of managing personal information accumulated on the desktop. 
The information overload problem, combined with the disconnection of data caused by application silos, is solved by the use of Semantic Web standards for storing and interlinking the desktop information. 
Data which before was stored separately by different applications can be now explicitly connected. The result is a network of interlinked personal information, which can be browsed by association, filtered and searched in a unified way. 

The Semantic Desktop overcomes the limitations of conventional desktops, where information is kept in different formats and application repositories, by using common vocabularies to describe the data, and a desktop-central place to store it, in a standardised format, accessible to all applications.

Indeed, the Semantic Desktop makes the information load manageable. However, new challenges emerge: one such new issue is how to design and realise semantic applications that use the infrastructure provided by the Semantic Desktop. We address this problem by dividing it into smaller, simpler challenges and providing solutions for each of them. To illustrate the solutions, we describe the design of a semantic note-taking application for the Nepomuk-KDE Semantic Desktop, called SemNotes. We use note-taking as an example because it is a desktop activity that is not limited to a specific domain, since the notes can widely vary in topic. It is also a common activity that plays an important role in Personal Information Management and that we believe would benefit from the use of semantic technologies.

\subsection{Challenges}
\label{sub:appchallenges}

In Section \ref{sub:question} we broke down the first research question \textbf{Q 1.}: \emph{How to build semantic applications and tools for the Semantic Desktop as to provide the best experience for the users, while creating reusable semantic data?}, into several sub-questions.
These are the challenges we found in designing new semantic applications for the Semantic Desktop:

\begin{description}
 \item[Q 1.1.] \emph{How to create semantic data that is correct, and maximises the reuse of existing Semantic Desktop data through interlinking?}\\
Applications should be aware of the data that is available on the desktop, from other applications. The data they produce is also accessible to other applications, and this should also be taken into account, because it raises the dual challenge of making sure that the data is represented correctly so that other applications can use it if they choose, as well as making the most of existing information through reuse and link creation. How to interlink information items is the most important challenge for adhering to the Semantic Desktop requirements of creating and maintaining a network of linked desktop data. It refers in the first place to the new information coming into the desktop through the application, and how to integrate it with the linked information that exists on the desktop. But it also covers the situation when the only new information created is in fact a new link between existing entities. 
Existing desktop data is heterogeneous. The reasons for this include the fact that different parts of it are created by different applications, with different functionalities and needs. So, although represented in a standardised way, the data is not always consistent, up-to-date or even correct. Furthermore, the data is also heterogeneous in terms of form: there are new types of information available to be integrated (i.e. tags, relations). Best practices advise the reuse of as much of the existing information as possible, because through reuse, the interlinking becomes deeper and richer. However, the possibility of reusing vast amounts of existing data raises other challenges, in selecting the right amount of necessary information for maximum benefit for the users, while not overloading them. The selection is complemented by the way the information is shown.
 \item[Q 1.2.] \emph{How to design the human-computer interaction in an application for the Semantic Desktop?}\\
This question relates to designing interfaces which support the existing workflow of the user, while integrating the additional semantic information in a useful way. A balance must be found between too much information, so that it interferes with the user doing tasks, and too little, or not enough to make a difference. The question also refers to more than just displaying the information in an interface --- allowing users to interlink information items easily, and generally aiding the creation of new semantic data is also a challenge for application development. Extra difficulty is added by the variety of interlinked information available on the desktop, combined with the reduced control of developers over what resources are linked from other tools.
 \item[Q 1.3.] \emph{How to correctly evaluate a semantic application?}\\
A challenge related to evaluation is the lack of a standardised dataset to use, due to the highly personal data required. Even if such a dataset existed, due to the fact that the applications on the Semantic Desktop are mainly related to personal information, it is difficult for participants to use the data provided, as they are not familiar with it. Evaluating PIM tools has been shown to yield best results when the test users are asked to perform their own tasks in their own set-up rather than attempting to simulate it with artificial tasks and artificial data. The reduced number of semantic application in each area makes it difficult to evaluate a new tool against an existing established one, thus the best candidates for a comparative evaluation are conventional (i.e. not semantic) applications with similar functionality. While possibly re-demonstrating that indeed linked data is more useful, comparing a semantic and non-semantic application requires well thought metrics, as the semantic features that need 
evaluating have no direct counterparts against which to evaluate them. For task based evaluations, it is difficult to find a common set of tasks that both applications can perform. A solution is to choose general, high level tasks, but that influences the granularity of the results.
\end{description}
