\section{Introduction}
\label{sec:sdwodintro}

The Semantic Desktop aims to enable better organisation of the personal information on our computers, by applying semantic technologies on the desktop. Just as Linked Data connects distributed data on the Web, creating a network of interlinked information, the Semantic Desktop connects personal data across application boundaries on the desktop, creating a network of personal information.
However, information on our desktop is often incomplete, as it is based on our subjective view, or limited knowledge about an entity.

On the other hand, the Web of Data contains information about virtually everything, generated by multiple sources, and theoretically unlimited. Connecting the desktop to the Web of Data would thus enrich and complement desktop information. Bringing in information automatically from the Web of Data would release the user from the burden of searching for information.

Connecting the two networks of information opens up the possibility of personal services on the desktop, which use external data but in the personal context of the user, highly connected to his personal data and focused on his interests. One such example is a service that finds implicit links between the publications that the user has on the desktop, and provides recommendations to other publications on the same topics, by the same authors, or related in another way. 
Another desktop service could use information from the Web of Data to notify the user of new concert dates in his area, based on the latest or most popular artists played on the desktop. 
Web data can also be used as a point of reference when working collaboratively, e.g., documents linked by the user to people, projects, or other resources from his semantic desktop can be shared together with the annotations, which can be accessed and reused outside of the Semantic Desktop where they were generated.

From the perspective of interlinking information, and using the frameworks provided by the Semantic Desktop and the Web of Data, we have separate islands of knowledge, both containing similar data, related to the same topics of interest to the user, but disconnected from each other. 

The disconnection appears in two forms:
\begin{itemize}
 \item The data on the desktop, although similar to that on the Web of Data, is described using specific \emph{desktop ontologies}, which are different from the ones found on the Web of Data. This schema mismatch makes interlinking data from the two datasets difficult.
 \item Identifiers (URIs) on the desktop are local to the desktop data space, they are not globally unique and cannot be dereferenced as normal Linked Data URIs are. Hence, it is hard to access and connect to local data from the Web.
\end{itemize}

To tackle this disconnection, it is necessary to create links between desktop identifiers and Web identifiers that refer to the same real-world thing.
This means we need to compare the data graph describing an entity on the desktop with the data graph of an entity on the Web. Leaving aside the use of different terminology within the data, the Web of Data is large, with billions of entities across hundreds of thousands of datasets. From this vast amount of information we must find and retrieve a relevant subset of entities, that are potential candidates with the desktop entity. Then we must decide if the candidates are similar enough with the desktop entity to create a link between the two. Because we wish to make the interlinking automatic, we must be able to decide with a high degree of precision which candidates among this subset are in fact referring to the same entity.

Our solution tackles the problems raised above by using a semantic search engine for the Web of Data, such as Sindice, to find and retrieve
a relevant subset of entities from the Web. We then present a matching framework, using a combination of configurable heuristics and rules to
compare data graphs, that achieves a high degree of precision in the linking decision. 

To evaluate our methodology with real-world data, we create a gold standard from relevance judgements by experts, and we measure the performance of our system against it.

Our solution proves that interlinking the two environments is feasible, and even more, it yields good results. Connecting desktop data with the Web enables the system to bring Web data to the users, instead of the users having to go find it by themselves.
