\section{Problem Statement}
\label{sec:problem}

The Semantic Desktop provides a solid framework for applications and systems to support the use of semantics on the desktop. However, having this framework is not enough, as the success and validation of any framework is reflected in its uptake and use. A key challenge that emerges is how to design and implement good semantic applications that use the infrastructure provided by the Semantic Desktop. This is a very vague and underspecified issue, that we divide into smaller problems for which we provide solutions. Important facets include presenting and visualising data, handling the relations among desktop entities, and incorporating new semantic data into the existing network.

The Semantic Desktop connects information in a network of linked personal data. Adding the, still mostly experimental, social aspect of sharing semantic personal data, and collaborating on it, we get clusters of networks of linked personal data. However, there is already a much larger network of Linked Data on the Web. The openly available Linked Data has grown extensively in recent years, as shown by the successive visual representations of the Linking Open Data cloud\footnote{\url{http://lod-cloud.net} Linking Open Data cloud diagram, by Richard Cyganiak and Anja Jentzsch.}. 
With such a vast network of Linked Data available, but unconnected to the desktop, the next step is to connect these two networks. This task is made easier by the fact that they share the same representation language, yet it is complicated by the different vocabularies used to describe the data, as well as the difference between the open world assumption of the Web versus the closed world assumption of the desktop. Building the bridge between these two domains raises several challenges, from ensuring privacy and security, to finding the best information in the large amount of available data.

\section{Research Questions}
\label{sub:question}

The above challenges are materialised into more precise research questions. At a general level, two questions will be tackled in the following chapters, each of them divided into more detailed sub-questions.

We start from the premise of the existing Semantic Desktop which solves the architectural issues, and gives a stable framework to build upon. However, it does not provide a complete ecosystem able to support the user in some of the more complex tasks. As such, we focus on two complementary directions:

Starting from the desktop environment, we look into:
\begin{description}
 \item[Q 1.] \emph{How to build semantic applications and tools for the Semantic Desktop to provide the best experience for the users, while creating reusable semantic data?}
\end{description}
And then we expand the scope from the desktop to the larger domain of the Web of Data:
\begin{description}
 \item[Q 2.] \emph{How to expand the scope of the Semantic Desktop into the realm of the Web of Data, to benefit the users and enhance their experience?}
\end{description}
We divide these questions into more specific sub-questions, that describe the facets of the issues tackled.

While building on top of the Semantic Desktop, there are additional challenges beside the ones raised by normal application design and development.

\begin{description}
 \item[Q 1.1.] \emph{How to create semantic data that is correct, and maximises the reuse of existing Semantic Desktop data through interlinking?}
\end{description}

Due to the blackboard type of architecture of the Semantic Desktop, applications should not be developed in a vacuum, but rather, they should be aware of other data that is available to them, and also be aware that the data they produce will be accessible to other tools. This raises the dual challenge of making sure the data is represented correctly so that other applications can use it, if they choose to, as well as ensuring the privacy and security of the data created. While is is impossible to say that information created is ever complete, applications must ensure that the data they create contains at least the minimum amount of information as to be usable by others. This requirement depends on the type of data created, but for new resources it includes basic properties such as a label to be used for display. Regarding the correctness of the data created, this requirement refers mainly to the way information is described according to the ontologies used, rather than being correct factually, which is a 
harder task and depends on factors outside the control of the application developers. Correctly described data contributes also to the last point of the question, enabling better discovery and reuse. 

The possibility of reusing vast amounts of existing data raises other challenges, in selecting the right subset of necessary information for maximum benefit, while not overloading the users with it. The selection is complemented by the way the information is shown:

\begin{description}
 \item[Q 1.2.] \emph{How to design the human-computer interaction in an application for the Semantic Desktop?}
\end{description}

The question refers to more than just displaying the information in an interface. Collecting input from the users, and generally supporting the creation of new semantic data is also a challenge. An extra difficulty is added by the heterogeneity of interlinked information available on the desktop, combined with the reduced control of developers over what resources are linked from other tools. 

While it has been demonstrated by several studies that semantic tools are better than non-semantic counterparts \cite{Sauermann2008,Franz2009} in supporting users in PIM tasks, evaluating semantic applications for the Semantic Desktop is still a difficult task to undertake. Therefore, the next question is: 

\begin{description}
 \item[Q 1.3.] \emph{How to correctly evaluate a semantic application?}
\end{description}

The challenge comes from the lack of a standard dataset on which to perform the evaluation. Even if such a dataset existed, due to the fact that the applications on the Semantic Desktop are mainly related to personal information, it is difficult for test participants to use the data provided, as they are not familiar with it. Evaluating PIM tools has been shown to yield more accurate results when the test users are asked to perform their own tasks in their own set-up, rather than attempting to simulate it with artificial tasks and data \cite{Kelly2006}. The reduced number of semantic applications in each area make it difficult to evaluate a new tool against an existing established one, thus leaving the alternative of evaluating against an existing non-semantic application. While possibly re-demonstrating that indeed linked data is more useful, comparing a semantic and non-semantic application requires well-thought metrics, as the semantic features that need to be evaluated have no direct counterparts against 
which to check them.

The second research question refers to connecting the Semantic Desktop to the Web of Data. The large amount of data available on the Web makes it difficult to find relevant information. An important part of connecting the Semantic Desktop to the Web of Data is finding common entities, like people or events, which might appear in both datasets. But entity resolution is hard when the input dataset is very large, and when only limited information is available for discriminating between similar entities. Therefore:

\begin{description}
 \item[Q 2.1.] \emph{How to find instances representing the same real-world thing described by a Semantic Desktop resource?}
\end{description}

Examples include persons, organisations, projects, events, and documents. Since both networks of Linked Data (on the desktop and on the Web) use the same representation, improvements in the graph matching algorithms help, but the two sides are unbalanced and most often use different vocabularies to describe the data.
Reconciling the desktop ontologies with the vocabularies used on the Web of Data requires mapping the ``few'' ontologies in the former category to the seemingly infinite number of vocabularies in the latter. 

\begin{description}
 \item[Q 2.2.] \emph{How to use the Web information which is related to a desktop resource?} 
\end{description}

The answer to the question depends of course on the application that uses the information and the desktop resource, but some points are general: retaining the provenance of each piece of information imported; finding and handling conflicting information and judging trust; deciding if and when should local information be deprecated and replaced with new Web information. 

Furthermore, connecting the Semantic Desktop to the Web of Data does not necessarily imply a unidirectional relation. A reverse link from the Web of Data to the Semantic Desktop should be considered as well. As always when making personal information available online, privacy concerns must be taken into account.

\begin{description}
 \item[Q 2.3.] \emph{How to make desktop data available online safely?}
\end{description}

While accessing desktop information from the Web is technically possible, in principle it goes against the idea of a closed personal Semantic Desktop. The suitable way of making desktop data available online is by publishing it, thus making it part of the Web of Data. 
Publishing information from the desktop is a simple enough task, but privacy issues arise when the data in question is highly interlinked with other private information.

Following the directions given by the two general questions, the approach we use focuses first on interlinking Semantic Desktop data through new semantic applications, and continues with linking the desktop to the Web of Data, through establishing new connections to the outside.
