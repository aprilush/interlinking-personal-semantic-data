\section{Visionaries}
\label{sec:visionaries}

The ideas behind the Semantic Desktop have been around for much longer than the actual devices (personal computers, netbooks, tablets or even smartphones) on which they run. The common problem the Semantic Desktops are trying to solve is finding better ways for knowledge workers to manage the ever-growing amount of information they need to work with and process. 
The most relevant historical precursors of the Semantic Desktop are, in chronological order:

Paul Otlet's Universal Bibliographic Repertory is a Belgian enterprise from the end of the 19th and the beginning of the 20th century, trying to set up a comprehensive system of all the bibliographic knowledge available, and keep it up to date as new knowledge appeared. It used a system of interlinked index cards, with reference codes pointing to relevant information, thus creating the first network of knowledge, although not an machine readable one. Despite not being widely recognised as such, it does represent the beginning of standardised knowledge and information management.

Vannevar Bush's Memex --- cited by most ``modern'' Semantic Desktops as inspiration, describes a first vision for a truly personal system for interlinking information. Although Bush's seminal paper ``As We May Think'' was published in 1945, his work on the Memex started well before that.

Douglas Engelbart's Augment --- built on the vision of Memex, the Augment was the first ``real'' functional system. As of 2012, it is still being used by its creator, despite being created in the early 1960s. Engelbart's ideas on bootstrapping are also relevant in the evolution of Semantic Desktops. 

Theodor Nelson's Xanadu --- imagined at roughly the same time as Augment, the Xanadu system was never completed. However, the ideas behind it inspired many of the future developments that shaped the Web and many technologies that exist today.

\subsection{Paul Otlet (1868 - 1944)}
\label{sec:otlet}

Although Bush, Engelbart and Nelson's works are usually cited as inspiration and background in most works about the Semantic Desktop, Paul Otlet's work is relevant through its influence on them. His work anticipated many of the features of the other historical systems described here.
Otlet, a Belgian lawyer turned bibliographer, was one of the (grand-)fathers of information science. He wrote numerous essays on the need for an international system for handling, indexing, linking, and accessing information, as a way of containing and managing the vast knowledge produced by man. Otlet created the Universal Decimal Classification (UDC) as the solution to the problems he described in his essays. The UDC pioneered the use of standardised index cards which allowed for continuous interlinking and referencing of entries, and represents one of the first faceted classification systems. The UDC was used to create a huge database called the Universal Bibliographical Repertory, which had 15 million entries by the late 1930's. The UDC is still in use throughout the world, although in revised and updated forms. It can be seen as precursor to high level taxonomies of concepts for classification, with notions of ``narrower" and ``broader" subject terms. 
Otlet's activity and vision are reflected in his chef d'oeuvre ``Trait\'{e} de documentation'' \cite{Otlet1934}. It is seen as precursor and motivator to the hypertext, mark-up languages and even the Semantic Web.

\subsection{Vannevar Bush (1890 - 1974) and the Memex}
\label{sub:bush}

Bush was an American scientist and public figure. During the World Wars he coordinated researchers from many domains, towards applying scientific advances to warfare and defence. He was an unofficial scientific advisor to two US presidents\footnote{Franklin D. Roosevelt and Harry S. Truman}, and led several institutions and committees, along with initiating and supporting the creation of the National Science Foundation in the United States. Throughout his career Bush passionately supported collaboration between researchers, which he saw as the fastest path to the progress of humanity. He went as far as to propose an exchange of scientific results between the United States and Russia during the Cold War, to promote collaboration and prevent the atomic bomb race.

\subsubsection{``As we may think'' (1945)}

Bush was deeply concerned with the continuation of the collaboration between scientists after the Second World War was over. Himself a scientist, Bush was also painfully aware of how the progress in science meant that more research results are published than can be followed, and that because of old and inadequate publishing systems, as well as the sheer amount of information, much of the new research is unreachable, lost or overlooked. 

In a lengthy essay titled ``As We May Think'' \cite{Bush1945}, published in The Atlantic Monthly and later in Life, Bush describes several devices, possible inventions of the future, as solutions to the problems of research publishing and as teasers of tantalising scientific advances. Some of the devices may seem amusing in the context of current developments, but looking back, they are remarkable through their accuracy and foresight. 
The most famous of the devices described by the article is the Memex --- ``... a future device for individual use, which is a sort of private file and library'', ``... a device in which an individual stores all his books, records and communications, and which is mechanized so that it may be consulted with exceeding speed and flexibility''. Leaving aside the physical description of the device, which was envisioned in the form of a desk with screens, drawers for storage, a keyboard, buttons, and levers, Bush has more or less described the functions of a present day personal computer --- a device where we store and consult personal documents and communications. Furthermore, the Memex, attempts to mimic the way associations in the brain works, by using ``associative trails'' of connected things. The trails can be created, modified, browsed, shared and collaborated on. ``Associative indexing'' was ``the essential feature of the Memex.'' 
Bush presents a greater purpose for his Memex --- creating new forms of linked (associative) encyclopaedias. They would serve as a record of the human knowledge accumulated over time and would ``accelerate technical progress''. 

Although the article was published in 1945, Bush had been working on the idea of the Memex for many years before, and although there are no clear indications of any influences from Otlet's work, their higher end goals are similar. 
Some claim that Otlet's work was indeed known to Bush, indirectly through Watson Davis' visit in 1932 and there is some overlap between the works of the two to support it\footnote{\url{edwardvanhoutte.blogspot.com/2009/03/paul-otlet-1868-1944-and-vannevar-bush.html}} \cite{Veith2006}.
However, Bush writes against indexing systems, that he sees as artificial and cumbersome, and one of the reasons why information is hard to find. But the Memex could benefit from something like Otlet's UDC, a common system for classification of the trails. The trail names used in the Memex are instead highly personal and cryptic, which can become a problem when it comes to sharing the knowledge \cite{Buckland1992}.

The elegant ideas behind the Memex influenced the fields of information science and information management, and the development of personal computing, hypertext and semantic technology. The Memex is the first Semantic Desktop described in the literature, and although it was never realised, it has influenced the works of both other two visionaries included in this section, Douglas Engelbart and Ted Nelson.

Bush continued to work on the project, and wrote further articles on the Memex, like ``Memex II'' from 1959, and ``Memex Revisited'' from 1967, which are included in the book \cite{Nyce1991}.

\subsection{Douglas Engelbart and Augment}
\label{sub:engelbart}

From a desire to ``improve the lot of the human race'' \cite{Goldberg1988}, Douglas Engelbart has devoted his career to ``augmenting the human intellect''. In pursuit of this goal, he invented many devices that greatly influenced the way we work with computers today: the mouse, the window, the word processor, video conferencing, remote procedure calls, hypertext, and more. However, all these notable advances were just first steps towards his greater purpose, that of achieving an augmented human intellect.

\subsubsection{``Augmenting the Human Intellect'' (1962)}

The 1962 report ``Augmenting the Human Intellect'' \cite{Engelbart1962} describes a conceptual framework for his research, presenting the details of the H-LAM/T system (Human using Language, Artefacts, Methodology, in which he is Trained). It defines four classes of possible augmentation means, through intertwining artefacts, language, methodology and training. It also discusses thought process and repertoire hierarchies, and how they influence problem-solving capabilities. Furthermore, Engelbart defines mental structures, concept structures and symbol structures, and their relationship to each other. He introduces the hypothesis that ``better concept structures ca be developed -- structures that when mapped into a human's mental structure will significantly improve his capability to comprehend and to find solutions within his complex problem situations.''

The report thoroughly quotes and discusses Bush's ``As We May Think'', which Engelbart believes ``to make a very convincing case for the augmentation of the individual intellectual worker.'' He focuses on Bush's description of the Memex, and highlights how the device fits within the conceptual framework of augmentation. He also shows how the associative trails pioneered by Bush map to the concept and symbol structures from H-LAM/T, and how they can be evolved further, in the context of a note-card system.
Apart from the high-level ideas that precede and build up to the ideas of the Semantic Desktop, the Augment report also describes and motivates the need for a well designed semantic model for such a system. In an exercise of forethought, Engelbart says that classification ``might be the most significant part of the work.'' He envisions that the extra effort required ``to form tags and links [...] consciously specifying and indicating categories'' would be rewarded by the capacity gained by the computer to understand and perform more sophisticated tasks. He describes the ontology creation model - establishing the categories and relations between them, dealing with and removing semantic ambiguity.

\subsubsection{``A Research Center for Augmenting Human Intellect'' (1968)}

The \cite{Engelbart1968} report describes the system that was shown at the 1968 Fall Joint Computer Conference in what has become known as ``the mother of all demos'' because of the size, novelty and sheer number of innovations shown together at one time. The demo, and report, are outstanding for many reasons --- the infrastructure required to run it, the number of people involved, the introduction of the mouse, the collaborative work environment, the first word processing software, and many more. However, from the Semantic Desktop point of view, the spectacular part comes from its realisation of the conceptual framework detailed in the 1962 report \cite{Engelbart1962} --- the introduction of explicit hierarchical structuring of information --- ``we adopted some years ago the convention of organising all information into explicit hierarchical structures, with provisions for arbitrary cross-referencing among the elements of a hierarchy''; ``the symbols one works with are supposed to represent a mapping of one'
s associated concepts, and further that one's concepts exist in a network of relationships.''
Allowing the creation of arbitrary links between elements of the hierarchy, and providing unique names for entities for better and faster access, represents the realisation of Bush's associative trails and makes Engelbart's On-Line System (NLS) the first functional predecessor of a Semantic Desktop. The semantics are limited to the network of statements in files, but this is an important first step towards deeper interlinking.

\subsection{Theodor Nelson and Xanadu}
\label{sub:nelson}

By his own account, Nelson was not an engineer but a philosopher and cinematographer, and ``a computer fan, computer fanatic if you will'' \cite{Nelson1974}. However, he saw the potential that computers had beyond data processing and computation. He envisioned a way of using the computer as ``an adjunct to creativity'' for writing and personal information management, in the style of Bush's Memex for personal use. 

Nelson foresaw the information overload crisis we are facing, and many of the developments that created it: the personal computer, enhanced communications, digital publishing, virtually infinite storage. He has imagined a solution to the problem ---  a ``psychic architecture'' of a system \cite{Nelson1973}, where psychic refers to ``the mental conceptions and space structures among which the user moves''. 
The idea was embodied in a system called Xanadu, the specification and requirements of which are detailed as early as 1965 in \cite{Nelson1965}. Xanadu transformed over time, from an ``effort not to forget'' and to organise personal information better, into a vision of unrestricted linking of passages of text.

Nelson proposes for Xanadu a new file structure called ELF (evolutionary file structure) made up of entries, lists and links, as a way or realising the associative trails of the Memex. The zippered list structure described by him allows ``any two entries to be connected'', by ``link-modes having different meanings to the user'' --- similar to how we use RDF in current semantic systems. 
Nelson shares with Engelbart the credit for the invention of hypertext (hypermedia and hyperfilm), where ``hyper'' has roughly the mathematical sense of ``extended, generalised, and multidimensional''. Hypertext's purpose was to reflect ``the real structure of the thoughts expressed'' by users.
The freedom to link anything with his version of hypertext was supporting what Nelson called transclusion, the process of including something by reference rather than by copying --- ``quoting without copying''. Already in his 1973 article \cite{Nelson1973} he mentions concerns for the security  of hyperlinked data, and based on transclusion, Nelson invented a new type of copyright management called transcopyright.
Zippered lists also have evolved into zzstructures, which are at the basis of ZigZag hyperstructure toolkit and the ZigZag personal environment \cite{Nelson2004}.

The Xanadu system was in development for many years, but it has never become fully functional. Despite it not being a success, the vision behind it has inspired many dedicated followers and influenced today's world of personal computing. 

\subsection{Influences and Influencers}

There is a distinguishable network of influence in the domain of networks of knowledge.

It is uncertain if Otlet's work has in any way influenced the inception of the Memex across the Atlantic at the same time, but the idea is supported by the fact that he ``reported enthusiastically on experiments with other bibliographical applications of technology, especially microfilm'' \cite{Goldschmidt1982,Rayward1990}. Furthermore, ``in the early 1930's Otlet began to speculate about how a wide range of then experimental technology --- radio, cinema, microfilm, and television --- could be combined to achieve a new complexity and variety of functionality in information searching, analysis, re-structuring and use'' \cite{Rayward1991}.

The influence of Bush's Memex on the works of Engelbart and Nelson is, however, undisputed, and stated by both in their writings. The directions were different though:  Engelbart NLS was designed by engineers, focused on their needs, and encouraging collaboration. Xanadu on the other hand was intended for personal use, for capturing the train of thought of one user and to serve as an extended memory. Some of the features envisioned by Nelson for Xanadu were incorporated in Engelbart's NLS --- the ability to link paragraphs of text, revise documents in real-time by moving pieces of text on the screen, preserve and track changes.
