\part{Prelude}
\label{part:prelude}

\chapter{Introduction}
\label{ch:introduction}

We accumulate massive amounts of information on our computers and other devices, so much that it becomes daunting and sometimes useless because it is impossible to remember it all: emails, appointments, tasks, documents, people, conversations, etc. In traditional desktop architectures this information is created and stored in various applications, each an isolated island of data, with its own storage and its own format, unaware of related and relevant data in other applications. But in our mind this information is not separate, rather everything is connected by implicit links. When we use information we do not work with just one detached slice of it, but we mentally follow connections to other pieces of information from multiple sources. Information is inherently interlinked and used in an integrated fashion.

The Semantic Desktop emerged as a solution to the data fragmentation and disconnection problem. It provides a foundation and a framework based on Semantic Web standards and technologies, for interlinking information on the desktop. Interlinked information is easier to manage and organise, and because it mimics users' mental models, it is easier to find when required, thus also offering a solution to the information overload problem. The Semantic Desktop overcomes limitations of conventional desktops by using a common representation for the data, common vocabularies to describe it, and a central place to store it. It creates a network of linked desktop data, in a standardised format, which is accessible to all applications.

Several implementations of Semantic Desktops exist, with varied function sets and capabilities. NEPOMUK is a project undertaking the task of creating a blueprint for the Semantic Desktop. 
The work presented in this thesis is grounded and built upon the framework provided by Nepomuk-KDE, a branch of the NEPOMUK project. 

The most common use case for the Semantic Desktop is Personal Information Management (PIM). However, having a relatively stable framework for the Semantic Desktop does not bring the search for improved PIM to an end, rather it uncovers a path to a solution. There is plenty of work ahead, exploiting and improving the framework based on experience, research and testing. This thesis focuses on providing new and improved means of creating and maintaining the network of interconnected personal semantic data, on the desktop and beyond.

\section{Problem Statement}
\label{sec:problem}

The Semantic Desktop provides a solid framework for applications and systems to support the use of semantics on the desktop. However, having this framework is not enough, as the success and validation of any framework is reflected in its uptake and use. A key challenge that emerges is how to design and implement good semantic applications that use the infrastructure provided by the Semantic Desktop. This is a very vague and underspecified issue, that we divide into smaller problems for which we provide solutions. Important facets include presenting and visualising data, handling the relations among desktop entities, and incorporating new semantic data into the existing network.

The Semantic Desktop connects information in a network of linked personal data. Adding the, still mostly experimental, social aspect of sharing semantic personal data, and collaborating on it, we get clusters of networks of linked personal data. However, there is already a much larger network of Linked Data on the Web. The openly available Linked Data has grown extensively in recent years, as shown by the successive visual representations of the Linking Open Data cloud\footnote{\url{http://lod-cloud.net} Linking Open Data cloud diagram, by Richard Cyganiak and Anja Jentzsch.}. 
With such a vast network of Linked Data available, but unconnected to the desktop, the next step is to connect these two networks. This task is made easier by the fact that they share the same representation language, yet it is complicated by the different vocabularies used to describe the data, as well as the difference between the open world assumption of the Web versus the closed world assumption of the desktop. Building the bridge between these two domains raises several challenges, from ensuring privacy and security, to finding the best information in the large amount of available data.

\section{Research Questions}
\label{sub:question}

The above challenges are materialised into more precise research questions. At a general level, two questions will be tackled in the following chapters, each of them divided into more detailed sub-questions.

We start from the premise of the existing Semantic Desktop which solves the architectural issues, and gives a stable framework to build upon. However, it does not provide a complete ecosystem able to support the user in some of the more complex tasks. As such, we focus on two complementary directions:

Starting from the desktop environment, we look into:
\begin{description}
 \item[Q 1.] \emph{How to build semantic applications and tools for the Semantic Desktop to provide the best experience for the users, while creating reusable semantic data?}
\end{description}
And then we expand the scope from the desktop to the larger domain of the Web of Data:
\begin{description}
 \item[Q 2.] \emph{How to expand the scope of the Semantic Desktop into the realm of the Web of Data, to benefit the users and enhance their experience?}
\end{description}
We divide these questions into more specific sub-questions, that describe the facets of the issues tackled.

While building on top of the Semantic Desktop, there are additional challenges beside the ones raised by normal application design and development.

\begin{description}
 \item[Q 1.1.] \emph{How to create semantic data that is correct, and maximises the reuse of existing Semantic Desktop data through interlinking?}
\end{description}

Due to the blackboard type of architecture of the Semantic Desktop, applications should not be developed in a vacuum, but rather, they should be aware of other data that is available to them, and also be aware that the data they produce will be accessible to other tools. This raises the dual challenge of making sure the data is represented correctly so that other applications can use it, if they choose to, as well as ensuring the privacy and security of the data created. While is is impossible to say that information created is ever complete, applications must ensure that the data they create contains at least the minimum amount of information as to be usable by others. This requirement depends on the type of data created, but for new resources it includes basic properties such as a label to be used for display. Regarding the correctness of the data created, this requirement refers mainly to the way information is described according to the ontologies used, rather than being correct factually, which is a 
harder task and depends on factors outside the control of the application developers. Correctly described data contributes also to the last point of the question, enabling better discovery and reuse. 

The possibility of reusing vast amounts of existing data raises other challenges, in selecting the right subset of necessary information for maximum benefit, while not overloading the users with it. The selection is complemented by the way the information is shown:

\begin{description}
 \item[Q 1.2.] \emph{How to design the human-computer interaction in an application for the Semantic Desktop?}
\end{description}

The question refers to more than just displaying the information in an interface. Collecting input from the users, and generally supporting the creation of new semantic data is also a challenge. An extra difficulty is added by the heterogeneity of interlinked information available on the desktop, combined with the reduced control of developers over what resources are linked from other tools. 

While it has been demonstrated by several studies that semantic tools are better than non-semantic counterparts \cite{Sauermann2008,Franz2009} in supporting users in PIM tasks, evaluating semantic applications for the Semantic Desktop is still a difficult task to undertake. Therefore, the next question is: 

\begin{description}
 \item[Q 1.3.] \emph{How to correctly evaluate a semantic application?}
\end{description}

The challenge comes from the lack of a standard dataset on which to perform the evaluation. Even if such a dataset existed, due to the fact that the applications on the Semantic Desktop are mainly related to personal information, it is difficult for test participants to use the data provided, as they are not familiar with it. Evaluating PIM tools has been shown to yield more accurate results when the test users are asked to perform their own tasks in their own set-up, rather than attempting to simulate it with artificial tasks and data \cite{Kelly2006}. The reduced number of semantic applications in each area make it difficult to evaluate a new tool against an existing established one, thus leaving the alternative of evaluating against an existing non-semantic application. While possibly re-demonstrating that indeed linked data is more useful, comparing a semantic and non-semantic application requires well-thought metrics, as the semantic features that need to be evaluated have no direct counterparts against 
which to check them.

The second research question refers to connecting the Semantic Desktop to the Web of Data. The large amount of data available on the Web makes it difficult to find relevant information. An important part of connecting the Semantic Desktop to the Web of Data is finding common entities, like people or events, which might appear in both datasets. But entity resolution is hard when the input dataset is very large, and when only limited information is available for discriminating between similar entities. Therefore:

\begin{description}
 \item[Q 2.1.] \emph{How to find instances representing the same real-world thing described by a Semantic Desktop resource?}
\end{description}

Examples include persons, organisations, projects, events, and documents. Since both networks of Linked Data (on the desktop and on the Web) use the same representation, improvements in the graph matching algorithms help, but the two sides are unbalanced and most often use different vocabularies to describe the data.
Reconciling the desktop ontologies with the vocabularies used on the Web of Data requires mapping the ``few'' ontologies in the former category to the seemingly infinite number of vocabularies in the latter. 

\begin{description}
 \item[Q 2.2.] \emph{How to use the Web information which is related to a desktop resource?} 
\end{description}

The answer to the question depends of course on the application that uses the information and the desktop resource, but some points are general: retaining the provenance of each piece of information imported; finding and handling conflicting information and judging trust; deciding if and when should local information be deprecated and replaced with new Web information. 

Furthermore, connecting the Semantic Desktop to the Web of Data does not necessarily imply a unidirectional relation. A reverse link from the Web of Data to the Semantic Desktop should be considered as well. As always when making personal information available online, privacy concerns must be taken into account.

\begin{description}
 \item[Q 2.3.] \emph{How to make desktop data available online safely?}
\end{description}

While accessing desktop information from the Web is technically possible, in principle it goes against the idea of a closed personal Semantic Desktop. The suitable way of making desktop data available online is by publishing it, thus making it part of the Web of Data. 
Publishing information from the desktop is a simple enough task, but privacy issues arise when the data in question is highly interlinked with other private information.

Following the directions given by the two general questions, the approach we use focuses first on interlinking Semantic Desktop data through new semantic applications, and continues with linking the desktop to the Web of Data, through establishing new connections to the outside.


\section{Approaches and Main Contributions}
\label{sec:contribution}

In order to answer the two research questions detailed above, the thesis describes two complementary directions for expanding the network of personal semantic data of the Semantic Desktop: creating connections within the desktop, and externally to the Web. 

The first direction, creating new semantic data inside the Semantic Desktop, including new connections between entities, focuses on enabling the users to do so, through semantic applications. We recognise the need for semantic applications designed specifically for the Semantic Desktop, as the framework provides capabilities above what the conventional tools can support. To demonstrate solutions to the challenges of developing semantic applications mentioned above, we describe the design process and implementation of SemNotes, a semantic note-taking tool. Although a relatively small application, it covers all the phases of the semantic data life-cycle as described in \cite{MoellerPhDThesis2009}, and does so in an unrestricted domain of use. We have written about developing SemNotes in \cite{Dragan2009a} and in more detail in \cite{Dragan2011a}.

The second, external direction of bridging the gap between the Semantic Desktop data and the Web of Data looks at capitalising on the common representation and structure of the data in the two networks. Because the two worlds are rarely disconnected, and a large part of the entities from the desktop appear online as well, we define and implement an algorithm which finds Web aliases for desktop resources. This part of the work initiated from linking publication information \cite{Groza2009} on the desktop and on the Web, and continued with a use case focused on publishing semantic notes as semantic blog posts without losing link annotations \cite{Dragan2010a}. The finalised algorithm and its evaluation are described in \cite{Dragan2011b}. 

To establish the background for the following work, we present a survey of Semantic Desktop systems and applications, which is an extended version of the one published as \cite{Dragan2012}.

\begin{table}
\centering
\ra{1.3}
\begin{tabular}{@{}lc@{\hs}l@{\hs}}
\toprule
Chapter & \phantom{a} & Research questions\\ 
\midrule

 \ref{ch:semnotes} Creating and Interlinking Semantic Data && \textbf{Q 1.1. Q 1.2. Q 1.3.}  \\ 

 \hfill on the Desktop with SemNotes && \\

 \ref{ch:sdwod} Bridging the Gap between the Semantic && \textbf{Q 2.1. Q 2.2.} \\

 \hfill Desktop and the Web of Data && \\

 \ref{ch:semblogging} Transforming Semantic Notes into && \textbf{Q 2.2. Q 2.3.} \\
 
 \hfill Semantic Blog Posts && \\

 \ref{ch:mischelperapps} Additional Extensions and Applications && \textbf{Q 1.1. Q 1.2. Q 2.1. Q 2.2.} \\

\bottomrule
\end{tabular}
\caption{The research questions tackled in each chapter.}
\label{tab:chaptersquestions}
\end{table}

\section{Thesis Outline}
\label{sec:outline}

The remainder of the thesis is structured as follows: 

\subsection*{Part \ref{part:background} --- Foundations}

We begin by setting the foundations for our work. This part is comprised of two chapters. 

\begin{description}
\item[Chapter \ref{ch:background}] describes the terminology and introduces some of the related work about the Semantic Web and its representative technologies, Linked Data, and Personal Information Management.
\item[Chapter \ref{ch:sd}] contains an extensive survey of existing Semantic Desktops, from the visionaries that inspired them, to discussing their architectures, and their applications, the common points and what they do differently. This chapter also describes in more detail the NEPOMUK Semantic Desktop, the framework on which this thesis is built on.
\end{description}

\subsection*{Part \ref{part:core} --- Core}

The core of the thesis presents our work, divided in four chapters: the first two reflect the approaches used to solve our two main research questions, and the last two contain respectively a combination of the approaches into one unified use case, plus a set of extensions complementing the work.

\begin{description}
\item[Chapter \ref{ch:semnotes}] describes our solution to the first research question \textbf{Q 1.} --- how to build semantic applications for the Semantic Desktop, and all of its sub-questions. To illustrate the solution, we present the design and implementation of SemNotes, a semantic note-taking tool for the Semantic Desktop, as an example application. We use note-taking as an example because it is a desktop activity that is not restricted to a specific domain, as the notes can vary widely in topic. It is also a common activity that plays an important role in Personal Information Management and one that we believe would benefit from the use of semantic technologies.
\item[Chapter \ref{ch:sdwod}] describes our solution to the second research question \textbf{Q 2.} ---  how to expand the scope of the Semantic Desktop into the Web of Data, by providing an algorithm, and its implementation, for finding Web aliases of desktop resources. The work is motivated by many of the resources from the Semantic Desktop, also appear on the Web, and the information available about them online could be used to augment the local data. By creating the connection between the local resources and their Web counterparts, we make the first step towards truly personalised desktop services using Web data, and also enabling the publication of desktop data safely online.
\item[Chapter \ref{ch:semblogging}] describes a system that uses of the work from the previous two chapters, publishing online the semantic notes taken with SemNotes, as semantic blog posts. Locally, the notes are well integrated and interlinked with the rest of the semantic network of desktop data. However, as the connections are to local resources,  publishing the note as a blog post directly would lead to broken links and the context of the note being lost. On the other hand, including the related desktop information as metadata with the blog post might lead to privacy issues. However, because the relevant desktop resources have Web aliases, we can replace the local links with Web links and preserve the context without the risk of exposing any private information.
\item[Chapter \ref{ch:mischelperapps}] contains short descriptions of several applications and extensions for the Semantic Desktop, which complement the main work presented before. We begin with extensions to SemNotes which were built to incorporate and test existing research in the area of Natural Language Processing, and which could be applied to note-taking, enhancing the user experience. We also describe the process of linking publication data with Sclippy, the tool that initiated our work on connecting the desktop with the Web of Data. Another application for the Semantic Desktop that could connect to the Web is Konduit. It was envisioned as a simple user friendly tool that allows users to create their own applications with semantic data, by visually defining workflows.
\end{description}

\subsection*{Part \ref{part:conclusion} --- Conclusion}

\begin{description}
 \item[Chapter \ref{ch:conclusion}] contains the conclusion of the work, reiterating the contributions and how they answer our research questions. We discuss some of our results and the insights we gathered from the work. We also outline ideas for future research.
\end{description}

