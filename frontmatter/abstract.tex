\section*{Abstract}

With the goal of improved Personal Information Management, the Semantic Desktop emerged as a solution to the problem of data fragmentation and disconnection on the desktop. It provides a framework based on Semantic Web standards and technologies, for interlinking desktop information. Interlinked data mimics the user's mental model, and as a result it is easier to manage, to find and refind, thus also offering a solution to the information overload problem. The Semantic Desktop overcomes limitations of the conventional desktop by using a common representation for the data, common vocabularies to describe it, and a desktop-central place to store it. Thus, it creates a network of linked desktop data, in a standardised format, accessible to all applications. 
However, having a relatively stable framework for the Semantic Desktop does not end the quest for improved Personal Information Management. 

To users, the Semantic Desktop framework is a transparent layer, and as such, its benefits must shine through the applications which use it. And so the challenge of designing \emph{good} semantic applications for the Semantic Desktop emerges. Important facets of this challenge include: presenting and visualising data, handling the relations among desktop entities, and incorporating new semantic information into the existing network of connected data. 
A second challenge is generated by the fact that the desktop, even a semantic one, is no longer sufficient for our information needs. The ever growing amount of online data, and the increasing volume of it which is available in structured form, enables further interlinking of personal semantic information with the Web of Data, to the benefit of the user.

In order to address these challenges, we follow the theme of interlinking personal semantic data in two threads through this thesis. First locally, within the Semantic Desktop, interlinking must be supported, and even more so \emph{encouraged} by semantic applications built on top of the framework. We describe the challenges of designing and developing a semantic application for the Semantic Desktop, and present our SemNotes as an example, detailing the design decisions made along the way. The second direction extends to bridging the gap between the Semantic Desktop and the Web of Data, through connecting matching entities from the two spaces. This bridge opens up the possibility of using the rich data from the Web to complement and augment a user's personal information in new and innovative ways. 